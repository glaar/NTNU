% latex article template

% cheat sheet(eng): http://www.pvv.ntnu.no/~walle/latex/dokumentasjon/latexsheet.pdf
% cheat sheet2(eng): http://www.pvv.ntnu.no/~walle/latex/dokumentasjon/LaTeX-cheat-sheet.pdf
% reference manual(eng): http://ctan.uib.no/info/latex2e-help-texinfo/latex2e.html

% The document class defines the type of document. Presentation, article, letter, etc. 
\documentclass[12pt, a4paper]{article}

% packages to be used. needed to use images and such things. 
\usepackage[pdfborder=0 0 0]{hyperref}
\usepackage[utf8]{inputenc}
\usepackage[english]{babel}
\usepackage{graphicx}
\PassOptionsToPackage{hyphens}{url}

% hides the section numbering. 
\setcounter{secnumdepth}{-1}

% Graphics/image lications and extensions. 
\DeclareGraphicsExtensions{.pdf, .png, .jpg, .jpeg}
\graphicspath{{./images/}}

\begin{document}
\pagenumbering{arabic}

\large{IT3010-Research Methods in IT, Assignment-1}

\normalsize
Magnus L. Kirø, \today


%idea: research into the performance and structure of system development teams. 

%Write an initial research proposal. The purpose of a research proposal is to
%convince others that you have a worthwhile research project. In this initial
%research proposal, you should include:

\section{Motivation}
This is research proposal. We purpose to look into team efficiency and
collaboration in regards to system development and IT. 

I propose to look into the issue of team collaboration an effectiveness win
system development. I want to find out what can improve team collaboration and
team self management/ self organization.    

The problem is important to the development process of information systems, and
other IT-systems.  

\section{Relevant articles}
%    a brief review of relevant articles (2-3 articles). 

\paragraph{Self-Organizing Roles on Agile Software Development Teams} 
Article link: \href{http://ieeexplore.ieee.org/xpl/articleDetails.jsp?tp=&arnumber=6197202&url=http%3A%2F%2Fieeexplore.ieee.org%2Fxpls%2Fabs_all.jsp%3Farnumber%3D6197202}{IEEE}

The objective of the article is to observe the and identify the roles of self
organizing teams. More specifically software development teams that uses the
Agile development methodology. 

The article identified six roles that manifest in the teams observed. Mentor,
coordinator, champion, translator, promoter, and terminator are the six roles
observed. 

The relevance of the article is the focus on how teams organize themselves and
how people take roles upon themselves to improve the performance of the team. 

\paragraph{Task Environment Complexity, Global Team Dispersion, Process
Capabilities, and Coordination in Software Development } 
Article link: \href{http://ieeexplore.ieee.org/xpl/articleDetails.jsp?tp=&arnumber=6583162&refinements%3D4291944822%2C4291944246%26ranges%3D2012_2015_p_Publication_Year%26matchBoolean%3Dtrue%26searchField%3DSearch_All_Text%26queryText%3D%28team%2C+system+development%29}{IEEE}

The article focuses on key aspects of software development in teams and study
their effects. Some important aspects are: rigor, standardization, agility, and
customizability. 

Distance and strictness of the team organization is found to have a negative
effect of the effectiveness of the team. Increased coordination is also a
negative factor. 

The relevance of this article lies in the aspects of interest. Are the
different aspects important, or are they not too relevant for self organizing
teams?

\paragraph{Coordination of Software Development Teams across Organizational
Boundary -- An Exploratory Study} 
Article link: \href{http://ieeexplore.ieee.org/xpl/articleDetails.jsp?tp=&arnumber=6613088&refinements%3D4291944822%2C4291944246%26ranges%3D2012_2015_p_Publication_Year%26matchBoolean%3Dtrue%26searchField%3DSearch_All_Text%26queryText%3D%28team%2C+system+development%29}{IEEE}

Look into the organization of teams across organization and geographical
barriers. 

Distance matters, even small ones. Project managers need to be aware of
differences in process between organizations. Formal processes and guidelines
should be established. 

Team collaboration across organizations is more of less the definition of how IT
consultants work today. Therefore this research, into how cross organization teams
work together, is relevant. 

\paragraph{Software team processes: A taxonomy} 
Article link: \href{http://ieeexplore.ieee.org/xpl/articleDetails.jsp?tp=&arnumber=6225952&refinements%3D4291944822%2C4291944246%26ranges%3D2012_2015_p_Publication_Year%26matchBoolean%3Dtrue%26searchField%3DSearch_All_Text%26queryText%3D%28team+efficiency%29}{IEEE}

The objective is to define the language, or set of words, able to describe team
interactions correctly. And to define a way of appropriately report team
collaborations.

The vocabulary can accurately describe team interactions, and help solve
communication breakdowns. 

The relevance of this article lies in the communications aspect. Where it
describes useful insight into how communication can be improved, and solved.

\section{Objectives}
%research objective(s). The research objective(s) should provide a simple, clear and concrete statement of purpose.
The objective of the research is to enlighten the area of system development
and the state of team organization and effectiveness. In detail, looking at how
best to organize and empower development teams in the IT-sector. 

\section{Research Questions}
%research question(s).  The research question(s) should be: 

%(a) answerable,
%(b) consistent with the research objective(s), 
%(c) adequate in term of scope and effort need to address the question(s), 
%(d) related to each other (if there is more than 1 research questions).

The following questions would be appropriate:

1: What is the state of team organization today?

2: How can effectiveness be improved in teams today? 

3: Can empowerment of development teams help efficiency of the development process
and product delivery?

\end{document}
This is never printed
