% latex article template

% cheat sheet(eng): http://www.pvv.ntnu.no/~walle/latex/dokumentasjon/latexsheet.pdf
% cheat sheet2(eng): http://www.pvv.ntnu.no/~walle/latex/dokumentasjon/LaTeX-cheat-sheet.pdf
% reference manual(eng): http://ctan.uib.no/info/latex2e-help-texinfo/latex2e.html

% The document class defines the type of document. Presentation, article, letter, etc. 
\documentclass[12pt, a4paper]{article}

% packages to be used. needed to use images and such things. 
\usepackage[pdfborder=0 0 0]{hyperref}
\usepackage[utf8]{inputenc}
\usepackage[english]{babel}
\usepackage{graphicx}
\PassOptionsToPackage{hyphens}{url}

% hides the section numbering. 
\setcounter{secnumdepth}{-1}

% Graphics/image lications and extensions. 
\DeclareGraphicsExtensions{.pdf, .png, .jpg, .jpeg}
\graphicspath{{./images/}}

% Title or header for the document. 
\title{
	Exercise 1 , TIØ4116 Microeconomics and Investment Science 
}
% Author
\author{
	Magnus L. Kirø
}
\date{\today}

\begin{document}
\maketitle
\pagenumbering{arabic}



\section{task 1}
\paragraph{a}
"ii.
Difference between demand and supplied quantities at a given market price. "
The market price increases with demand, as supply decreases. 

\paragraph{b}
"i.
Luxury goods "
Imported, and not life essential. 

\paragraph{c}
"ii.
A big reduction in the crude oil refinery cost. "
The upward swing in the supply curve comes from the fact that we can produce
more oil at a lower cost, making the profits greater. Same production results
in greater income, or the same amount of expenses generates more oil. 

\paragraph{d}
"iii.
Increased price-demand elasticity for fresh meat "
Because people will buy meat when prices are low and store it for later, making
the demand for meat sporadic and drawn out over time.

\section{task 2}
A technological innovation will change price over time. 
Not necessarily at once, but in the long term. 
Depending on the innovation for aluminium and plastic, we can see price
increase of price decrease. Together with increased supply or demand. 

\url{https://upload.wikimedia.org/wikipedia/commons/7/7a/Supply-and-demand.svg}
The graph shows a standard supply-demand graph. It shows that as demand
increases from d1 to d2, we can follow the increase in supply. And from the
supply see the increased price and quantity of sold goods. This is a possbile
outcome for the change in steel price. 

We cannot be sure of price increase or decrease unless we know something about
the innovation. Without more information we cannot make a sure statement about
the price change of steel. If the innovation increases strength in plastic and
aluminium the demand for those would probably increase, accompanied with the
price. Which means that demand for steel is decreasing and we have to reduce
the price to sell the same amount of steel. But we could also increase the
price on steel because there is not enough supply of aluminium or plastic. 
This kind of price change is a trade-off. Either we lower the price to increase
volume and keep profit. Or we lower volume to increase price and keep profit. 

\section{task 3}

\paragraph{a}
What price policy should the financial director rec
ommend? 
A slight reduction of price to 0.9 would be a good way to increase profit while
the capacity doesn't increase to much. With price of 0.9 we can estimate that
we have 47.5 person kilometre per year, and an income of 42.75 million. This
reduction in price will result in slight increase in capacity, wile the profits
also increase. 

The current price is good. The capacity would not have to be raised, while
we still have quite a good income. 

The important considerations are the operation cost per person kilometer, which
decreases per additional kilometer. And the ticket price per person kilometer,
which can be reduced when there are more people kilometer. The main problem is
resource optimisation. Where we need to find the optimal point of profit for
the minimum of transport cost.  

No, a price reduction to 0.60Kr would not make sense. The capacity is to low,
which would increase costs in other departments. A slow and steady increase
of demand and capacity should be followed with a reduction in price.

\paragraph{b}
\url{https://en.wikipedia.org/wiki/Arc_elasticity}

e = \% change in x / \% change in y

(30, 40)/ (1.2, 1): 28.5 / 18.2 = 1.57

(40, 55)/ (1, 0.8): 31.5 / 22.2  = 1.42

(55, 65)/ (0.8, 0.6): 16.6 / 33.3 = 0.5

The connection between elasticity and the ticket price is that while the price
decreases and the kilometre increases the elasticity decreases. We get less
flexibility as we increase capacity and decrease price.  

\paragraph{c}
Drink prices are higher than food prices.
One part of it is that people will drink something while eating. So the demand
for drink is higher than the demand for food. And therefore the price is
bigger. 

Also the food prices are more competitive due to quality of the food and
location of the place. 

As for branded clothing. People pay for the social symbol it is to wear the
right brand. The social norms drive the prices of some brands because the brand
is cool or trendy, which increases demand and the price.   

\end{document}
This is never printed
