% latex article template

% cheat sheet(eng): http://www.pvv.ntnu.no/~walle/latex/dokumentasjon/latexsheet.pdf
% cheat sheet2(eng): http://www.pvv.ntnu.no/~walle/latex/dokumentasjon/LaTeX-cheat-sheet.pdf
% reference manual(eng): http://ctan.uib.no/info/latex2e-help-texinfo/latex2e.html

% The document class defines the type of document. Presentation, article, letter, etc. 
\documentclass[12pt, a4paper]{article}

% packages to be used. needed to use images and such things. 
\usepackage[pdfborder=0 0 0]{hyperref}
\usepackage[utf8]{inputenc}
\usepackage[norsk]{babel}
\usepackage{graphicx}
\PassOptionsToPackage{hyphens}{url}

% hides the section numbering. 
\setcounter{secnumdepth}{-1}

% Graphics/image lications and extensions. 
\DeclareGraphicsExtensions{.pdf, .png, .jpg, .jpeg}
\graphicspath{{./images/}}

% Title or header for the document. 
\title{
	IT på frivillig basis, omorganisering, prioritering og kommunikasjon.
}
% Author
\author{
%	Magnus L Kirø \\
%	IT-sjef, Studentmediene i Trondheim
}
\date{\today}

\begin{document}
\maketitle
\pagenumbering{arabic}

\begin{abstract}
Abstrakt, kort definisjon av innholdet og funnen av denne oppgaven \ldots
\end{abstract}

\section{Introduksjon}
Innlede oppgaven og definere kontekst. 

\paragraph{Kontekst}

\paragraph{Problemstilling}
presentere problemstillingen. 

\paragraph{Struktur}

\section{Tidligere Arbeid}\label{tidligere arbeid}
Det arbeid som er lagt ned tidligere for å bygge opp under denne oppgaven. 

\section{Frivillige organisasjoner}
\section{Omorganisering}
\section{Kommunikasjon og prioriteringer}
\section{Konflikter}
it-drift. 

\section{Diskusjon}\label{resultater}
Diskutere de funn og problemer som er presentert i de tre tidligere seksjonene. 

\section{Konklusjon}\label{konklusjon}
Oppsumering av de funn og konklusjoner vi kan dra fra de konflikter og 

\section{Referanser}
De kilder jeg har brukt of referert til i oppgaven. 

\section{Notater}

studnr_kandnr

har vi lært noe?
skjønt pensum?
lært pensum?
fått med oss noe?

mest refleksjons oppgave for min egen del. 

koble erfaring mot teori. 

aggregert erfaring + teori om ledelse. 

konflikt gjennom forventmingsavklaring. 

bytte roller og se ting fra foskjellige vinkler. 

12pt, 1,5 linjeavstand, 15-20 sider.

seksjoner: 
innledning, teori, metodedel, diskjusjon(størst), avslutning. 

teori: 
finne og lese teorier og ledelses filosofer. 
må være i sammenheng med tema. 

analyse/diskusjon er det viktigste:
få inn motsetninger og erfaringer. Ting som er lært og åpenheten for å lære noe
nytt. 

refleksjonsoppgave. (sånn har jeg oppfattet det)



\end{document}
This is never printed
