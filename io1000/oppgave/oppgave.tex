% latex article template

% cheat sheet(eng): http://www.pvv.ntnu.no/~walle/latex/dokumentasjon/latexsheet.pdf
% cheat sheet2(eng): http://www.pvv.ntnu.no/~walle/latex/dokumentasjon/LaTeX-cheat-sheet.pdf
% reference manual(eng): http://ctan.uib.no/info/latex2e-help-texinfo/latex2e.html

% The document class defines the type of document. Presentation, article, letter, etc. 
\documentclass[12pt, a4paper]{article}
%12pt, 1,5 linjeavstand, 15-20 sider.
%Filnavn: 'studnr_kandnr.pdf'
% 713703_10015.pdf

% packages to be used. needed to use images and such things. 
\usepackage[pdfborder=0 0 0]{hyperref}
\usepackage[utf8]{inputenc}
\usepackage[norsk]{babel}
\usepackage{graphicx}
\usepackage[options]{natbib} % bibliography thing.
\usepackage{hyphenat}
\PassOptionsToPackage{hyphens}{url}

% hides the section numbering. 
\setcounter{secnumdepth}{-1}

% Graphics/image lications and extensions. 
\DeclareGraphicsExtensions{.pdf, .png, .jpg, .jpeg}
\graphicspath{{./images/}}

% Title or header for the document. 
\title{
	IT på frivillig basis, omorganisering, prioritering og kommunikasjon.
}
% Author
\author{
%	Magnus L Kirø \\
%	IT-sjef, Studentmediene i Trondheim
}
\date{\today}

\begin{document}
\maketitle
\pagenumbering{arabic}

\begin{abstract}
Oppgaven tar for seg aspekter ved omorganisering, de kommunikasjons
utfordringene man har der og litt om konflikter rundt det. Frivillige
porganisasjoner er tema kontekst for oppgaven og ledelsesteori står sentralt i
diskusjonene.  
\end{abstract}

% listings
\tableofcontents
%\addcontentsline{toc}{section}{List of Contents}
\clearpage

\renewcommand{\baselinestretch}{1.50}\normalsize
\section{Introduksjon}
%Innlede oppgaven og definere kontekst. 
%refleksjonsoppgave. (sånn har jeg oppfattet det)

%\paragraph{Kontekst}
%\hspace{0pt}\\
Et sentralt tema for erfaringene beskrevet senere er omorganisering. Hovedsaklig
bygging av ny struktur og grunnlegging av kultur. Tematisk sett er ledelse,
omorganisering, frivillighetskultur, og kommunikasjon sentrale.  

Erfaringen som danner grunnlaget for oppgaven er ervervet gjennom en
omorganiseringsprosess i en frivillig organisasjon. Den motvilje og det
byråkrati som er konsekvens av omorganiseringen spiller en viktig rolle.
Ringvirkningen av omorganiseringen har påvirket arbeidet i organisasjonen og i
enkelte avdelinger. En avdeling i en vanskelig situasjon krever konkrete
avgjørelser og konstant kamp mot endring der endring ikke har noen hensikt.  

I en omorganiseringsprosess er kommunikasjon veldig viktig. I frivillige
organisasjoner må man balansere tidsbruken. Man kan ikke bare kommunisere uten
å få tid til å få utrettet noe også. Balansen mellom møter og eget arbeid er
vanskelig når man driver frivillig. Spessielt med studenter.

Prioriteringer rundt dette er også et utfordrende område. Når alle vil at man
skal gjøre som de sier må man velge etter eget hode. Man kan ikke oppfylle
alles ønsker til enhver tid. Problemer rundt prioriteringer og begrunnelse for
gitt prioritering kan til tider være vanskelig og i noen tilfeller
konfliktskapende.     

%\paragraph{Problemstilling}
%\hspace{0pt}\\

Som tittelen på oppgaven sierbeskriver er ordlyden på problemstillingen
følgense: "IT på frivillig basis, omorganisering, prioritering og
kommunikasjon.". 

Problemstillingen tar for seg de utfordringer og egenskaper man må ha, eller
må tilegne seg, for å starte opp og drive en IT-avdeling på frivillig basis. 
Diskusjonen rundt tema dreier seg om noen nøkkelegenskper en leder på ha
for å kunne utfylle stillingen sin effektivt. 

Formålet med oppgaven er å belyse og reflektere over, de innvirkningene og
effektene, omorganisering, prioriteringer, og kommunikjasjon har i en frivillig
organisasjon.  

%\paragraph{Struktur}
%\hspace{0pt}\\
Oppgaven består av syv deler. 'Introduksjonen' som definerer den konteksten vi er
i. 'Teori' delen som presentere de teorier som er nødvendige å ha med for at
resten av oppgaven skal gi mening. 'Frivillige organisasjoner' som sier en del
om organisasjoner bassert på frivillighet og de problemene man har der.
'Omorganisering' er en samling observasjoner fra en omorganiseringsprosess i en
frivillig organisasjon. 'Kommunikasjon' tar for seg kommunikasjons aspekter ved
ledelse og generelt. 'Konflikter' går inn på noen vanlige årsaker til
konflikter og hvordan man kan håndtere dem. Og til slutt har vi
'Oppsummering' med noen av de hovedaspektene som er nevt i oppgaven. 

\section{Teori}\label{tidligere arbeid}
Gruppers utvikling er en viktig del av det å skape en ny kultur. Man har de
fire fasene som er beskrevet av Bruce Tuckman, orientering, utprøving, norming,
og performing. De forskjellige fasene danner forskjellige egenskaper og
relasjoner innad i gruppen. 
\cite[]{teamet}

Orienteringsfasen er den fasen som tilrettelegger det å bli kjent. Man møter
gruppen og man finner de ulikhetene og kunnskapene man har. Det er en slags
kartlegging av kompetanse, interesser, og personligheter i gruppen. Naturlige
ledere bemerket i denne prosessen. 

I utprøvingsfasen utfordrer parter i gruppen hverandre og man legger lista for
vidre sammarbeid. Man tester grenser og prøver å finne sin plass i gruppen.
Interessekonflikter og knuffing mellom medlemmer er veldig vanlig i denne
fasen. På mange måter bryter man ned de fordommer medlemmene har, før man
iverksetter gjenoppbygging i normeringsfasen. 

Normeringsfasen tar for seg utgjevningen av konflikter og normalisering av
arbeidsforholdene. Det er her man kommer til en slags enighet og hvordan ting
skal fungere og hvordan man vil ha det i arbeidsmiljøet. Som leder er det mest
her man legger føringer for hvordan gruppen skal arbeide vidre. Skapningen av
en felles kultur og nedleggelsen av felles verdier skjer her. Man endrer fokus
på gruppens energi til de problemene man vil at gruppen skal løse, slik at de i
arbeidsfasen kan ha full fokus på synergi. 
\cite[]{teamet}

Arbeidsfasen tar for seg synergiskapningen. Altså hvordan man finpusser på
sammarbeidet til gruppen og løser problemer. Rett og slett den delen av
gruppearbeidet hvor man har satt konflikter til side og fokuserer på de felles
målene man tidligere har blitt enige om. Graden av måloppnåelse i denne fasen
er høy, men det er ikke alltid en gruppe kommer til arbeidsfasen. Det er mye
som kan skje på veien hit. I mange tilfeller endrer man gruppens sammensetning
før man kommer til arbeidsfasen, eller man har fått nye eksterne påvirkninger
som endrer gruppens dynamikk.  

Vidre har vi tid som et styrende element. Joseph McGrath mener at man ikke
bruker de fire, tidligere beskrevede, fasene sekvensielt, men at grupper har
fire moduser som byttes på etter behov. De fire modusene er oppstart,
problemløsning, konfliktløsning, og utføring. Modusene har mange likhetstrekk
til de tidligere nevnte fasene. 

Det strides om hvorvidt alle modusene må være tilstede i et gruppearbeid eller
ikke. Alle modusene er viktig for gruppens utvikling, men ikke nødvendigvis for
gruppens oppnåelse av resultater.  

William Schutz sier med FIRO-teorien at alle grupper trenger tre faktorer:
nærhet, kontroll, og åpenhet \cite[]{teamet}. Forskjellige medlemmer av en
gruppe trenger forskjellig grad av de tre faktorene. Noen trenger kontroll,
mens andre trenger åpenhet. Personlighet spiller en viktig rolle i hvordan man
tilpasser arbeidsmiljøet til disse faktorene. Noen personer trenger mer
tilretteleggelse og oppfølging enn andre. Faktorene er viktige i utviklingen av
en gruppe.

For å beskrive balansen mellom forskjellige elementer av gruppesammensetningen
har vi SPRG-modellen. Den beskriver samspillet mellom kontroll, omsorg,
opposisjon, avhengighet, synergi, og tilbaketrekning. Synergi og
tilbaketrekning er to indikatorer på en gruppes kvalitet. Kontroll,
omsorg, opposisjon, og avhengighet er fire gruppefunksjoner som beskriver
gruppens interne tilstand. Gruppens plassering i SPGR-rommet basserer seg på
disse seks konseptene, og vil variere mye eller lite avhengig av eksterne
påvirkninger. 

En gruppes rollefordeling forteller om gruppens status. Det vil si at man ved å
se på hver person sin påtatte rolle kan si noe om tilstanden til gruppen som
helhet. Hvis man har mange klart definerte roller har man en lavt utviklet
gruppe. Om rollefordelingen flyter og er vanskelig å finne har man en godt
utviklet gruppe. Det hele handler om fri flyt. Om gruppen har fri flyt av
roller og ansvar har man en velutviklet gruppe med gode forusetninger for å
løse problemer og skape resultater \cite[]{teamet}.  

I dannelses fasen av grupper er balanse mellom kontroll og omsorg viktig. For
mye kontroll og man ender opp med en splittet gruppe med konkurranse innad. Med
for mye omsorg og for lite kontroll vil gruppen ikke ha noe håndfast å forholde
seg til og dermed vil gruppen gå i oppløsning grunnet for lite produktivitet.

Den ideelle balansen mellom kontroll og omsorg i frivillighet er kombinasjonen
av selvstendig arbeid med predefinerte rammer og sosiale tiltak som binder
gruppens medlemmer sammen gjennom omsorg og sosiale kontrakter. 
De predefinerte rammene for arbeid sammen med oppfølging utgjør kontroll. 
Ansvar for egen læring er et godt eksempel på idealet, med gitte rammer skal
man utvikle seg selv. Ulempen er at gruppens modenhet bestemmer læringsevnen,
jo mer moden en gruppe er, jo større læringsutbytte er det.     

En kompliserende faktor for ledelse og gruppedannelse er medlemskap i flere
grupper. Med forskjellig kultur og formål i forskjellige grupper innad i samme
organisasjon får man problemer med produktiviteten på et eller annet tidspunkt.
Noen grupper ved, spessielle omstendigheter, utvikler seg negativt og blir
mindre produktiv som resultat av utviklingen. Noen grupper ender også opp med å
være selvdestruktie, altså de går i oppløsning når utviklingen er veldig
negativ, eller har vært negativ over lengre tid. \cite[]{teamet} 

Problematisk for en gruppes integritet og effektivitet er ekstern innblanding, fra
personer eller andre faktorer. Personer har en begrenset kapasitet til å
samhandle med andre. I mange tilfeller trenger man tid til å tilpasse seg selv
og sin rolle i en gruppe. Med tilpassningen og tilvenning til sin rolle får man
mer selvtillit og etterhvert mer energi til å håndtere eksterne innspill.
Etterhvert som det blir kjedelig innad i ens egen gruppe vil man vende blikke
utover og interagere mer med eksterne de andre gruppene rundt seg.   

Som leder tar man på seg forskjellige kapper i forskjellige situasjoner. Man
trer inn i formelle og uformelle roller. I en del tilfeller er det viktig å
påpeke dette konkret gjerne verbalt, men veldig ofte gjennom ritualer. De
formelle rollene er de stillingene eller vervene man har påtatt seg, og de
ansvarsområdene for følger stillingen man har. De uformelle rollene er den
sosiale personen eller rollen man velger å ta forskjellige situasjoner. Ved
bevisst å endre roller, i formelle og uformelle situasjoner, kan man påvirke
andres roller i situasjoner og dermed fremheve eller nedtone denne personen
eller personens adferd. Slik kan man forbedre en gruppe ssosiale dynamikk. 

Gitt en struktur vil det være naturlig i en del tilfeller at ens formelle rolle
er tilbakelent eller trilbaketrukket. Dette tilsier ikke at man ikke har
innflytelse på gruppen i den gitte situasjonen, men heller det at man har mer
innflytelse når man først bruker det innflytelsen man har. \cite[]{teamet} 

\section{Frivillige organisasjoner}
%koble erfaring mot teori. 

Organisasjoners struktur blir fort stor og kompleks. I en middels stor
organisasjon, 100 til 400 ansatte, har man ofte behov for en godt definert
organisasjonsstruktur. Dette medfører ofte ekstra kompleksitet. Den økte
kompleksiteten vil være fremtredende i kommunikasjon og sammarbeid i
organisasjonen.

Når man ser på strukturen som frivillig organisasjon blir kompleksiteten av
strukturen fremtredende. Samhandling blir i større grad påvirket og mange i
organisajonen vil ikke investere i det å sette seg inn i strukturen. Noe som
igjen vil gå på bekostning av effektiviteten til organisasjonen som helhet. 
Som leder i en frivillig organisasjon er en stor del av jobben å sette seg inn
i strukturen og vite hvem som har ansvaret for hva og hvem som har kunnskap på
hvilke områder. 

Som del av en organisasjon er det naturlig å ha en IT-avdeling, eller kontakt
med IT på en eller annen måte. Når man har egen IT-avdeling som driver med
utikling kommer det med sine egne utfordringer, men også med en del goder.
Som sjef for en frivillig IT-avdeling får man erfaring med mange aspekter av
en organisasjon og de utfordringene organisasjonen skaper. 

IT på frivillig basis har sine egne spilleregler, slik som et hvert fagfelt har
sine type mennesker, sine normer og sine steriotypier. Man har sine egne
viktige verdier, og sitt eget syn på hva som er viktig for organisasjonen. Og
med det kommer store utfordringer med prioriteringer av funksjonalitet og
fokus.    

Som leder av en avdeling i en frivilling organisasjon er balansen mellom
frivillig arbeid og andre beskjeftigelser en problematisk affære. Man må til
enhver tid prioritere mellom avdelingen man er leder for, den overordnede
gruppen man er medlem av og de verdiene og den strategien organisasjone har
lagt til grunn for arbeidet man utfører. 

Frivillige organisasjoner kontra andre ogranisasjoner sliter med
prioriteringene til medarbeiderene. Man må hele tiden motivere folk til å
prioritere sitt frivillige verv. De insetiver som er vanlige i arbeidslivet er
ikke tilgjengelig for frivillige organisasjoner. Kanskje med unntak av
forplaining og sponsing av sosiale arrangementer. 

I den frivillighetskulturen man har i Trondheim må man være konkurranse dyktig
blant organisasjonen. Man må ha de beste vilkårene for å få de beste folkene og
holde på dem over tid. Man må på mange områder også legge seg på et nivå som
kan konkurrere med de bedriftene som holder bedriftspresentasjoner for
linjeforeningene også. 

Utfordringene dette medfører må man overkomme ved å ha
et bedre sosialt tilbud og et godt tilbud for faglig utvikling. Som leder må
man tilrettelegge disse forholdene godt og skape læringssynergier for sine
medlemmer. Når man leder en frivillig organisasjon må man finne ut hva som
driver folk til å være med og sørge for at disse verdien blir ivaretatt. Sterkt
fokus på faglig utbytte og sosialt gode forhold skaper et godt sosialt lag, som
igjen skaper sterke sosiale kontrakter. De sosiale kontraktene kan på mange
måter ses på som den kulturen man bygger opp over tid. 

Kulturen i den frivillige organisasjonen er det viktigste elementet for å ha en
god organisasjon. Ved å ha en god kultur vil folk trekke til organisasjonen og
man forteller venner og kjente om hvor bra det er. Slik får man nye søkere go
nye flinke medarbeidere. Dette er viktige aspekter som man tar i betraktning
som leder. 

Et av hovedproblemen ved de frivillige organisasjonene i Trondheim er at
utskiftningen av medarbeidere er veldig høy. Man har folk engasjer bare et,
kanskje to år. Dette er lite når man må bygge kulturer og tunge systemer over
tid. Som leder på man finne nye folk to ganger i året, og integrere dem i
gruppen når de er blitt tatt opp i gjengen. 

\section{Omorganisering}
%koble erfaring mot teori. 

Under en omorganiseringsprosess tar man, enklet sagt, de normer og verdier man
har fra før av og kaster ut vinduet. Man dropper den kulturen man har fra før
og begynner i mange tilfeller helt på nytt. Dette er et kjempeproblem for
mange. I en slik omorganiseringsprosess må man finne de verdiene man vil ta
vare på og dyrke de gjennom prosessen. 

Prosessen som legger grunnlag for erfaringen her reduserte effektiviteten til
IT-avdelingen veldig. Det var også mye kunnskap og en del prosjekter som har
lidd igjennom prosessen. Vi endte også opp med å starte en hel del prosjekter
på nytt. Spessielt for omorganiseringsprosess var sammenslåingen av fire
eksisterende IT-grupper.

I utgangspunktet lå IT-avdelingen et semester forran resten av organisasjonen i
sammenslåingsprosessen. Men det ble gjort dårlig arbeid med integreringen i den
perioden, tross tidligere godt arbeid. Dette ser man etterdønningen av enda. Og
ganske sikkert et års tid til, men det er vanskelig å si og for mange vanskelig
å se.

Erfaringsmessig er omorganiseringsprosesser veldig lærerike. Man får mye ny og
god informasjon ut av det, og man kommer ut av prosessen med en hel rekke ting
man ville gjort annerledes. Det grunnlaget man danner i prosessen må man bygge
på etterpå. Resultatet er at man får en base å bygge vidre på, men som ikke er
veldig funksjonell.  

Som leder i en omorganiseringsprosess må man hele tiden tenke over de
konsekvensene som er ved de valgene man gjør. De tingen man velger på fokusere
på får konsekvenser for hele organisasjone, og man må over tid tilpasse
prioriteringene.   

I utviklingen av en IT-avdelingen som gruppe er det mange utfordringer. Man har
gruppedynamikken om de rollene forskjellige medlemmer tar på seg for å få ting
til å gå rundt. Starten presenterer gruppens medlemmer for hverandre og man
prøver å bygge relasjoner for å skape en fungerende gruppe. Over tid får man
tilpasset gruppen og skapt synergier. Lederen tilrettelegger for at folk skal
bli kjent of finne ut hva de er gode på mens de lærer seg ny teknologi og får
faglig utbytte.  

Det er spessielt utprøvingsfasen som er viktig. Her prøver man forskjellige
måter å jobbe på og man ser hva som fungerer. Det å lede en gruppe igjennom
denne fasen kan til tider være problematisk. Når man får sterkt eksternt press
vil denne utprøvingsfasen og opposisjonsmodusen til gruppen forsterkes. Man kan
observere sterke tendenser til redusert effektivitet, og til tider kommer man
så langt at det kan kalles en skikkelig konflikt. 

Problematisk er det spessielt når slike konfliker er bassert på manglende
kunnskap på tvers av avdelinger i organisasjonen. Disse kunnskapsgapene jobber
men som leder aktivt med å tette. Mangelen på kunnskap gjelder ikke bare den
ene parte, men begge.  

De eksterne gruppene rundt IT-avdelingen har presset og ikke fått de
resultatene de håpet på. Dette skyldes dannelsen av avdelingen og den kulturen
og gruppedynamikken som må være tilstede før man kan utsette gruppen for større
ekstern påvirkning. Typisk driver man med team-building i arbeidslivet for å
sammkjøre konsulent-team før de kommer ordentlig i gang med oppdrag. For
organisasjonen som helhet er dette en forsinkende faktor de ikke får gjort noe
med. Forståelsen av dette, eller mangelen av forståelse, skaper irritasjon og
negative påvirkninger utad fra avdelingen til resten av organisasjonen.     

\section{Kommunikasjon}
%koble erfaring mot teori. 

I en ideell situasjon forbedrer man kommunikasjonen over tid. Man blir bedre på
å formulere seg, og man blir bedre til å tolke andres ønsker. Det man i
hovedsakk blir bedre på er å forstå andres problemstillinger og ønsker. Uansett
kommunikasjonsform er det viktig å bli enige om noe man skal gjøre. Er man
enige om den neste handlinge er det lettere å forholde seg til noe senere.
Ellers kommer man tilbake til situasjoner hvor en part tror man har kommet til
enighet om noe, hvor man ikke har det.

Møter som kommunikasjonsform er i mange tilfeller et godt alternativ. Man får
sagt mye til mange, og man får den sosiale interaksjonen som danner sosiale
kontrakter. Men man må ikke glemme at all enighet er bort når man forlater
rommet. For å bøte på det problemet må man alltid sikre seg møtereferater og
oppsummeringer av møter man ikke har vært på.

Effektivitet i møtevirksomhet er også et aspekt hvor mange kan forbedre seg.
Ordstyring og dedikert referent er gode strategier for dette. Ellers blir det
ofte slik at den naturlige lederen i rommet må steppe inn og ta styringen. Der
det ikke er en naturlig leder får man veldig ofte uenighet og veldig lite
effektiv oppfølging. Et alternativ for å effektivisere slike møter kan også
være å ta inn en ekstern konsult, en megler eller tillitsperson, som kan styre
samtalen og holde fokus på tema.

For å øke produktiviteten i en gruppe er det også naturlig å bytte roller. Man
vil gjerne ta på seg rollen som ordstyrer når man har mange meninger, mens man
trekker seg tilbake om saken som diskuteres ikke angår en. Det er også viktig
at alle gruppens medlemmer prøver å bytte på rollene etterhvert, slikt av
forskjellige parter belyser et problem fra forskjellige sider. Rolle bytte
utgjør en vital forskjell i kommunikasjonen i møter.   

Et aspekt av ledelse er det å sile informasjon. Man presenterer den
informasjonen man mener er viktig for sin avdeling. Det er ofte en god strategi
å filtrer bort de elementene man ikke har noe interesse av selv. De man er
leder for har antagelig ikke bruk for den informasjonen heller. Denne formen
for sensur er bra fordi det gir større rom for fokus hos den enkelte
medarbeideren.

Erfaringsmessig er det veldig mange som ikke bryr seg om organisasjonen som
helhet, men kun er interessert i sitt lille hjørne av samensuriet de mener er
resten av organisasjonen. For disse medarbeiderene er det viktig å påpeke og
fremheve informasjon som faktisk er viktig for dem. Slik som endringer i
strategien og verdiene som ligger til grunn for den jobbe de gjør. Nærmere
bestemt om premissene for deltagelse i gruppen endres bør alle vite om det med
en gang.

Med god kommunikasjon på tvers i organisasjoner øker man produktivitete ved at
man eleminerer kommunikasjonsledd. Det er ikke noe vits for en avdeling å gå
opp et nivå i hierarkiet, for så å gå ned igjen et annet sted. Det skjer
uheldigvis ofte at man får redusert kommunikasjon og dårlige resultater fordi
man ikke får tak i rett person når man trenger det.  

En ulempe ved å aktivt filtrere informasojnsflyt ettersom man ser behoved for
det er at man skaper en polariseringe i organisasjonen. Man har de som vet hva
som skjer og bryr seg, samtidig som man har de som ikke vet og kanskje ikke
bryr seg. Dette er i mange tilfeller veldig frustrerende for ledelsen, da de
ikke kan være sikre på at den viktig informasjonen har kommet ut til alle
parter.  

Kommunikasjon rundt prioriteringer er vanskelig. Interessepartene i et prosjekt
må i mange tilfeller prøve å se hele bildet, mens man holder fokus på det gitte
prosjektet. Ressurs fordelingen blant prosjekter og omfordeling av ressurser
skjer ofte med innspill fra øvrig ledelse. I mange tilfeller er det gunstig å
ta avgjørelser i fellesskap, men i en del tilfeller er det mer effektivt og
lønnsomt for kunnskapspersonen å ta avgjørelsen.  

De manglende kunnskapene skaper kommunikasjonsgap som er vanskelig å overkomme,
og man vil over en tid ha vansker med å få utrettet noe fornuftig. Det er
viktig å praktisere god kommunikasjon, slik at man kan utvikle seg på det
området over og forhindre miskommunikasjon på sikt. Det er også et håp om at
man gjør seg godt forstått og får skapt sammarbeid på tvers i organisasjonen
over tid.  

Vanligvis tas avgjørelsen, om prioriteringer av de forskjellige prosjektene og den ønskede
funksonaliteten, demokratisk mellom de interessehaverene som er med i
kommunikasjonsprosessen. I mange sammenhenger skjer det ikke slik, men heller
at den ene parten presenterer ønsket funksjonalitet og handling, mens den andre
parten kjapt bestemmer seg for å utføre de oppgavene som gagner en selv best.
Dette skaper også konflikter i organisasjonen.  

Når man nærmer seg en konflikt eller et sårt tema er det vanlig at den ene
parten går inn i opposisjonsmodus. Parten er ikke villig til å sammarbeide
fordi man er redde for at de oppgaver og prioriteringer man får servert ikke er
de beste prioriteringene man kan ta. 

\section{Konflikter}
%koble erfaring mot teori. 
 
Konflikter er ikke til å komme utenom. De dukker opp bare man ser rart på noen.
Og i organisasjoner kan det ofte ende med konflikt i tilfeller hvor man
ikke helt vet hvem som har ansvaret for hvilke ting. Dette er et vanlig
senario. Noen blir sinte og irritert på noen andre, mens en tredje part har
ansvaret for den tingen. Med dårlig kommunikasjon ender men opp med en
situasjon hvor noen er sure på noen andre uten god grunn.

Ressursfordeling er også en sterk grunn til konflikt. Man ønsker ofte at sine
egne behov blir prioritert over andres. Og det kan hende man ikke skjønner
hvorfor ens egne behov ikke er viktigere enn andres. Som leder må man her være
god på å si at ting er som de er og at man har fokus på de tingen man har fokus
på av gitte hensyn. Om man forklarer situasjonen til folk vil de ofte forstå
det litt bedre enn de gjorde før. 

Erfaringsmessig kommer konfliktene ofte av avhengiheter eller
kommunikasjonsproblemer. De vanskeligste konfliktene er de som basserer seg på
avhengigheter. Det er vanskeligere å forstå at den tingen man vil ha utrettet
først kan skje etter et gitt antall andre oppgaver. Dette er spessielt
vanskelig på tvers av fagfelt. Her kommer opplæring inn i bildet som et viktig
aspekt. 

I frivillige organisasjoner er opplæring en av de aktivitetene som tar lengst
tid. Og gjerne det som skaper flest unødvendige konflikter. Man forventer at
nå som man har 5 nye folk så går ting så mye fortere. Men det gjør det ikke før
om et semester når de nye medarbeiderene kan noe. Kommunikasojnen av at ting
tar tid er vanskelig i omgivelser hvor forventningene er urealistiske. 

Forventninger som kilde til konflikter er vanskelig å gjøre noe med. Man må
stadig fortelle folk at vi ikke jobber så kjapt som man tror. Og når man
sammenlikner seg selv med noen som gjør det samme på proffersjonell skala vil
man alltid komme til kort med forventningene.      

Problematisk i konfliktløsning er når partene finner ut hvor problemet ligger,
og så frasktiver man seg ansvaret ved å påpeke at det er noen andre som har
ansvaret for saken det gjelder. I slike tilfeller er det viktig at
ansvarsfordelingen er klart definert på forhånd. Og at eventuelle endringer
blir dokumentert fortløpende.  

Det som værre er, og som skaper flere konflikter, er de som påtar seg oppgaver,
men som ikke fullfører eller følger opp gitt oppgave. Dette skaper kaos of
konflikter i resten av organisasjonen. Eneste måten å løse slike problemer er å
omdelegere oppgavene.   

\section{Oppsummering}\label{konklusjon}
%aggregert erfaring + teori om ledelse. 
Roller og rollebytte er et gdt verktøy som leder. Man får anledning til å
påvirke grupper på forskjellige måter, og man får sett ting fra forskjellige
vinkler. Det er viktig å sette seg inn i saker fra andre synsvinkler sin
hovedrolle. Forskjellige roller oppnår forskjellig reksjon i forskjellige
miljøer.

For å ha god kommunikasjon må man holde en god tone, og være forberedt til
møter man har. Man kan også benytte hersketeknikker og kropsspråk for å påvirke
andre parter. Det viktigste er at man trener på det å kommunisere godt og kan
annerkjenne sine feil når man blir konfrontert med det. Man må ikke være redd
for å ta kontakt. 

For konflikter og håndtering av dem må man være godt forberedt til
løsningsprosessen. Det lønner seg å høre på alle parter og se deres side før
man kritiserer noen. Og så lenge man holder fokus på målet og ivaretar ønskene
til alle partene kommer man godt ut av konflikter. Mange konflikter er bassert
på misforståelser.  

I store organisasjoner må man vite om de ulike avdelingen og kompetansepersonen
for å kunne kommunisere og lede effektivt. Man må ivareta gruppers integritet
og man må erkjenne at man ikke er ekspert på alt for at samspillet skal
fungere. Kommunikasjon på tvers av avdelinger er også viktig for å eliminere
byråkratisk arbeid og unødvendig tunggroddhet.  

Gruppens modenhet påviker gruppens produktivitet. I noen tilfeller vil det også
være gunstig at gruppens produktivitet går på bekostning av produktiviteten.
Det er mest fordi man prioriterer gruppens langtidsutsikter overfor gruppens
kortsiktige resultater. Når man skaper en ny kultur må man ha fokus på de
interne faktorene først og sørge for at man får en godt fungerende gruppe før
man kan begynne å løse problemer for andre på en effektiv måte.  

Gjennom omorganiseringsprosesser lærer man masse om gruppers dynamikk og
samhandling. Man får innsikt i organisasjonens struktur og virkemåte samtidig
som man får intim kjennskap til de verdiene medarbeiderene hadde som grunnlag
tidligere. 

%Bibliography
\addcontentsline{toc}{section}{Referanser}
\bibliographystyle{plainnat}
\bibliography{bibliography}

\end{document}
This is never printed
