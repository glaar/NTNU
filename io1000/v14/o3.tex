% latex article template

% cheat sheet(eng): http://www.pvv.ntnu.no/~walle/latex/dokumentasjon/latexsheet.pdf
% cheat sheet2(eng): http://www.pvv.ntnu.no/~walle/latex/dokumentasjon/LaTeX-cheat-sheet.pdf
% reference manual(eng): http://ctan.uib.no/info/latex2e-help-texinfo/latex2e.html

% The document class defines the type of document. Presentation, article, letter, etc. 
\documentclass[12pt, a4paper]{article}

% packages to be used. needed to use images and such things. 
\usepackage[pdfborder=0 0 0]{hyperref}
\usepackage[utf8]{inputenc}
\usepackage[english]{babel}
\usepackage{graphicx}
\PassOptionsToPackage{hyphens}{url}

% hides the section numbering. 
\setcounter{secnumdepth}{-1}

% Graphics/image lications and extensions. 
\DeclareGraphicsExtensions{.pdf, .png, .jpg, .jpeg}
\graphicspath{{./images/}}

% Title or header for the document. 
\title{
	Ledelse i Praksis, Øving 3. Konflikthåndtering. 
}
% Author
\author{
	Magnus L Kirø \\
	IT-sjef, Studentmedien i Trondheim AS. 
}
\date{\today}

\begin{document}
\maketitle
\pagenumbering{arabic}

Alle svarer på 2, 3, 4 og 6. Du velger selv om og eventuelt hvilke
spørsmål du vil vektlegge mest men besvarelsen bør være på 2

1.
Nevn typiske konflikter som kan oppstå mellom medlemmer eller mellom deg som
leder og andre i gjengen/gruppen din. 


Ta med årsaker og typiske forløp og legg vekt på
det som er mest relevant for deg.

2.
Har du noen råd til deg selv og gruppen din til hvordan (slike) konflikter kan
unngås? 

Er medarbeidersamtaler relevant i denne sammenhengen? 

Hvor langt har det gått før du vil kalle det en konflikt og hvor lenge vil du 
vente for å ta opp problemet?

3.
Dersom en situasjon allerede har oppstått, hvordan vil du gå frem for å ta opp
dette med vedkommende? 

Ta med hvordan vil du organisere samtalen og hvem vil du involvere.

4.
Dersom konflikten har eskalert til et helt uakseptabelt nivå, hvilke virkemidler 
har du og hvordan kan du gå frem? 

Hvordan kan du være helt sikker på at den eller de det gjelder har forstått
alvoret,det vil si at de har samme oppfatning som deg om hvor alvorlig det er?

Hva har det å si for resten av gjengen/gruppen hva du gjør og hvordan du
opptrer i denne sammenhengen?

5.
For deg som har vært nødt til å takle en konflikt i gjengen din: Hvordan synes
du selv at du taklet situasjonen? 

Ta med om du synes du kunne gjort noe annerledes, om den kunne
vært unngått, om det er noe du er fornøyd med og hvordan konflikten og
håndteringen av den påvirket deg.

6.
Kapittel 17 i pensumboken Teamet omhandler konflikt. Etter å ha lest kapittelet
beskriv hvilke ideer du fikk for hvordan du kan løse konflikter i gjengen din eller i eksemplene
du har beskrevet i foregående spørsmål. Referer til pensum.

\end{document}
This is never printed
