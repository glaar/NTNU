% latex article template

% cheat sheet(eng): http://www.pvv.ntnu.no/~walle/latex/dokumentasjon/latexsheet.pdf
% cheat sheet2(eng): http://www.pvv.ntnu.no/~walle/latex/dokumentasjon/LaTeX-cheat-sheet.pdf
% reference manual(eng): http://ctan.uib.no/info/latex2e-help-texinfo/latex2e.html

% The document class defines the type of document. Presentation, article, letter, etc. 
\documentclass[12pt, a4paper]{article}

% packages to be used. needed to use images and such things. 
\usepackage[pdfborder={0 0 0},colorlinks=true,linkcolor=blue]{hyperref}
\usepackage[utf8]{inputenc}
\usepackage[english]{babel}
\usepackage{graphicx}
\usepackage[top=3cm, bottom=3cm, left=2cm, right=2cm]{geometry}
\PassOptionsToPackage{hyphens}{url}
\usepackage{setspace}

% hides the section numbering. 
\setcounter{secnumdepth}{-1}

% Graphics/image lications and extensions. 
\DeclareGraphicsExtensions{.pdf, .png, .jpg, .jpeg}
\graphicspath{{./images/}}

% Title or header for the document. 
\title{
	The uses and limitations of game theory in psychology
}
% Author
\author{
	%Magnus L. Kirø \\
}
\date{\today}

\begin{document}
\onehalfspace
\maketitle
\pagenumbering{arabic}

Course code: PSY2012

Semester: Autumn 2014

Candidate number: 10077

Course responsible: Ruud van der Weel

\tableofcontents
\newpage 

\section{Literature and research}
% The uses and limitations of game theory in psychology

%Describe the difficulty of finding articles. 
Finding relevant game theory articles linked towards psychology proved to be
difficult. Searching with simple queries, such as 'game theory' or
'game+theory+ psychology', proved to give poor results. This implies that
research into the field of game theory in psychology must be a diverse field,
or a very sparse one.

By taking the few articles that resulted from the search the impression of the
research field begins to form. Based on the Paul A. Wagner (2013)
\cite{psyinvest}, the impression of the field becomes wide. Wagner writes about
game theory as a means of investigation. He describes a wide area where game
theory is used, and elaborates on the basics of it. Wagner states that
"Together game theory and psychology excavated into mental life more than
previous behavioral stipulations and methodologies would have ever
allowed."\cite{psyinvest}. 

%a brief discussion of why there are still psychologists working in these
%fields(game theory related approaches.)

For social scientists game theory can be a new tool to look at human
motivations for decision making. More specifically weighting of wanted goals
and risks, and a wider range of mechanisms are provided by game theory in a
psychology context. As game theory provides wider possibilities for research it
is likely that the use of game theory in psychology will emerge in new ways
over time. Per now it looks like it will take some time for all parts of
the community to embrace game theory, but some already have. Rather game theory
lacks momentum toady, and would benefit from more use.   

\section{Game Theory}
%short explanation of what game theory is
Game theory is shortly described as strategic decision making. It is in essence
interactive decision making. Zero-sum games, such games where one persons
winnings equals the other players net losses, was addressed early by game
theory. Developing game theory from zero-sum games it is now used in wide
range of behavioral contexts. Game theory has evolved into a umbrella term that
today consists mostly of the logical aspects of decision making.\cite{wiki}

Three assumptions was proposed by von Neumann and Morgenstern. Game theory
builds on these three basic facts as a foundation for further assumptions. The
three assumptions are: humans are self-interested, humans are rational, and
humans are self-determining consumers.\cite{psyinvest}

To the basic assumptions there were added five more. These are: 
1: All outcomes can be known to varying degrees of certainty, 2: Player
information is often incomplete, 3: Utilities(measures of one's relative gain)
can be measured, 4: The utilities of all outcomes when the other assumptions
are met can be discounted and summarised in a single quantity, 5: Some games
are competitive, some are constant sum games, and some constant sum games and
non-zero sum games often favor dominant strategies whose equilibrium invites
cooperation as an attractive strategy \cite{psyinvest}.

PD, the prisoners dilemma, is a classic example of a game theory. It describes
the scenario where two prisoners are interrogated in separate rooms without
contact. And they shall try to get the best deal with the police. Either can
give up the other, or be silent. This gives four outcomes. Prisoner A tells on
prisoner B, prisoner B tells on prisoner A, both tells on each other, and none
of them tell. By applying rewards, or in this years in prison, to the option
we have that if only one of them tells on the other, the one that tells gets 0
years in prison, the other one gets 10. If both tell they get 5 years each, but
if none of them tell they get 2 years each.\cite{psyinvest}

The described PD scenario bottoms out in a paradox when we look at the
assumptions of game theory. Both prisoners are self-interested and will
probably choose to tell on the other prisoner, resulting in a 5 year sentence
for each. But this means that the 2 year for each option is unreachable, which
is the paradox. 

By adding assumptions about the context of the game, such as that the prisoners
know each other, then the game have changed and other options open up. Now it
seems that securing the best decisions is no longer a simple mathematical
problem, but rather more complex problem involving psychology. When this was
presented a new direction opened up for science to explore.\cite{psyinvest}

Nash and other found out that, given a zero-sum or constant-sum game, there was
a set of actions that would result in an equilibrium for the expected value for
all players. This happens when every player has found an acceptable expected
value in a common strategy. Such a strategy will dominate all other strategies.
If such a strategy would be found in society, world peace and social problems
would be reduced to simple puzzles. But that is hardly the case in the real
world. \cite{psyinvest}

Game theory has a wide range of application.
"Not only is game theory used in economics, international trade, military 
strategizing, and business operations at every level it is now also used to
illuminate various evolutionary models in biology, anthropology, sociology, and
psychology all in addition to economic theory."\cite{psyinvest}.
As an example it is observed that wealthy people find cost of lottery tickets a
poor buy, while the risk reward picture looks totally different for the poor. 

An aspect to risk vs reward problem, or maximization of reward is the 'Beauty
Game'. The beauty game is a problem where every man wants the pretties girl in
the group, but no one can have her. If every one goes for the pretties girl,
only one can be successful, but if everyone goes for a different girl in the
same group the chances are increased for everyone. Another approach to this
problem would be to remove oneself from the first round, and aim for the
greater expected value in a discrete second round.\cite{psyinvest} 

On the contrary an experiment found that strangers who were offered money to help
move a couch was less likely to help than strangers not offered a reward. The
experiment shows that rewards can dampen the response form people. People seem
to value money in a way that increases possibilities in life, while it is
weighted against some value assigned to labour, or personal
dignity.\cite{psyinvest}  

As for cooperation, which is a central aspect of game theory and reward
maximization, Scharlemann et al. (2001) looked at how people perceive the value
of smiles. This is good example of how game theory has elaborated on the social
sciences. The article talks about how smiles effect the cooperativeness of
people. Participants score pictures of varying smiling degree. The main
variable Scharlemann et al. (2001) are interested in is the ability to identify
cooperative partners. They do this by examining the value of a smile in a
simple bargaining context.\cite{smile}   

Two research questions are stated by the paper, 1: 'Does smiling elicit trust
among strangers?', 2: 'Is there a difference between the sexes in assessing
trust?'. As signalers, people who smile, always benefit form smiling, there has
to be some way for the receiver to verify the authenticity of the smile. This
results in a handicap, or cost, for signaling false information. Non-smiling
individuals may be discriminated by receivers due to hidden bad qualities about
that individual.\cite{smile} 

Scharlemann et al. (2001) conclude the article by answering the research
questions, and briefly presenting the results. The experiments find that both
men and women are more trusting towards the opposite sex, Smiling positively
affects trust, and that facial features can affect cooperation regardless of
smiling. Smiling then serves as an informative stimulus which promotes trusting
behaviour. While smiles are increasing trust, then genuine smiles should have a
greater effect than false ones. Face-to-face interactions are also found to be
better than anonymous interactions.\cite{smile}

% humans are self-interested
% humans are rational
% humans are self-determining consumers 

Based on the three base assumptions, stated earlier, there are a few things
that limits game theory. Humans are not always rational, and humans are not always
self-interested.

To take these statements a bit further. Humans are often steered and influenced
by emotions. Hunger makes people eat. Such feeling based necessities are
rational acts in a way, but thoroughly anchored in underlying feelings beyond
the rationale. The use of game theory in social settings will for most people
prove difficult. Game theory states that the strategies used are rational, but
people performing them might be incapable of executing them over time, thereby
acting irrational.   

Self-interest is the focus on the desires and need of oneself. A simple
argument that people does not act in self-interest all the time are mothers.
Mothers in vary many cases acts in the interest of her children. The whole
concept of family to some degree disregards the self-interest that is required
by game theory. In a family context the members of the group works for the
betterment of the group, and not necessarily itself. 

\section{Conclusion}\label{conclusions}
%We worked hard, and achieved very little.

%What has been discussed in this essay. 
This essay has covered the basics of game theory, the prisoner dilemma, Nash
equilibrium, and an example of modern use of game theory. Literature search and
the state of game theory has been touched. And some limitations of the base
assumptions of game theory has been discussed. 

% applications of game theory. 
The applications of game theory are more numerous now than ever, but the
diversity has made it difficult to narrow the field of search for game theory.
Meaning that while game theory is being used more, the use of it is also more
hidden, and maybe of a more inspirational influence rather than of a direct
influence. 

Concluding statement that Game theory is more important than ever, but it is
hard to find articles relating directly to game theory. Although it might be a
lot easier to find new, and relevant, research if a more specialised field
under the umbrella term that game theory has become. Another thing to do would
be to do an extensive literature review of game theory and its sub fields of
research.  


% bliography
%\cite{deyer}
\addcontentsline{toc}{section}{References}
\begin{thebibliography}{1}
%\bibitem{deyer}
%Dyer, J. H., Kale, P., & Singh, H. (2001). Strategic alliances work. MIT Sloan
%Management Review, 37-43.

\bibitem{psyinvest}
Paul A. Wagner. Game Theory as Psychological Investigation. InTech, 2013
\url{http://cdn.intechopen.com/pdfs/43924/InTech-Game_theory_as_psychological_investigation.pdf}

\bibitem{smile}
Jörn P.W. Scharlemann, Catherine C. Eckel, Alex Kacelnik, Rick K. Wilson. The
value of a smile: Game theory with a human face. Journal of Exonomic Psychology
22 (2001) 617-640. 

\bibitem{wiki}
Game Theory on Wikipedia, 30.11.14,
\url{https://en.wikipedia.org/wiki/Game_theory}

\end{thebibliography}

\end{document}
This is never printed
