% latex article template

% cheat sheet(eng): http://www.pvv.ntnu.no/~walle/latex/dokumentasjon/latexsheet.pdf
% cheat sheet2(eng): http://www.pvv.ntnu.no/~walle/latex/dokumentasjon/LaTeX-cheat-sheet.pdf
% reference manual(eng): http://ctan.uib.no/info/latex2e-help-texinfo/latex2e.html

% The document class defines the type of document. Presentation, article, letter, etc. 
\documentclass[12pt, a4paper]{article}

% packages to be used. needed to use images and such things. 
\usepackage[pdfborder=0 0 0]{hyperref}
\usepackage[utf8]{inputenc}
\usepackage[english]{babel}
\usepackage{graphicx}
\PassOptionsToPackage{hyphens}{url}

% hides the section numbering. 
\setcounter{secnumdepth}{-1}

% Graphics/image lications and extensions. 
\DeclareGraphicsExtensions{.pdf, .png, .jpg, .jpeg}
\graphicspath{{./images/}}

% Title or header for the document. 
\title{
	Målsetninger, Øving 1, Ledelse i Praksis
}
% Author
\author{
	Magnus L Kirø \\
	IT-Sjef, Studentmediene i Trondheim 
}
\date{\today}

\begin{document}
\maketitle
\pagenumbering{arabic}

\section{SmitIT}
%(1) Fortell kort om gjengen/seksjonen din. Hva er gjengens/seksjonens funksjon(er) og
%oppgave(er)?

SmitIT: Studentmediene i Trodnehim, IT. IT-avdelingen. Vi jobber med utvikling av IT-systemene som blir brukt i Studentmediene. 

Avdelings funksjon er å tilrettelegge og tilby løsninger for produksjon og publikasjon av medie innhold i Studentmediene i Trondheim. Vi gjør hverdagen til journalister, teknikkere og administrasjonen letter ved å tilby og tilpasse smarte teknologiløsninger. I bunn og grunn gjør vi det mulig for resten av organisasjonen å fungere.  

\section{Utfordringer}
%(2) Hva har tidligere vært noen av din gjengs/seksjons utfordringer. Hvilke utfordringer
%står dere overfor i dag?
Fra tidliger(før sammenslåingen) har jeg vært Maskinist(IT-sjef) i Under Dusken. Da var hovedproblemet integreringen med resten av gjengen. IT var lite synlig og jobbet sjeldent sammen med resten av gjengen. Det er også utfordrende at de aller fleste i organisajonen ikke ser hva IT gjør eller hvorfor de gjør det de gjør. Litt av problemet er at IT overhodet ikke syns når de har gjort alt som de skal. Vi er stort sett synlige når ting ikke fungerer som det skal. 

Synligheten er en utfordring som har blitt vanskeligere i ny organisering. Avdelingen har blitt større og vi har slått oss sammen med Student-Media. Denne overgangen har gjort at IT har måttet fokusere på seg selv og bli integrert oss i mellom først, før vi kan ta steget ut i resten av organisasjonen. 

Utover synligheten er hovedutfordringen sammkjøring av systemer. Når fire tidligere organisasjoner slår seg sammen samtidig og alle IT-systemene skal slås sammen sier det seg selv at det ikke er en enkel prosess. Samkjørings-prosessen er også en tidkrevende prosess. Dette gjør at det tilsynelatende ser ut som IT ikke er så veldig effektive. IT ser framgangen, men det gjør ingen andre. Enda vanskligere blir det når vi ikke har noe konkret å vise til på møter med resten av ledelsen heller. 

Kort oppsummert: Sosial tilhørighet innad i IT og systemtilpasningen for ny organisering er utfordringene vi står overfor. 

\section{Personlinge Målsetninger}
%(3) Hva er dine personlige målsetninger i rollen som gjengsjef/styremedlem? Hva er
%gjengens målsetninger for det kommende året? Hva vil du/dere for å nå disse målene?
\paragraph{Målsetningene} mine kan deles inn i personlige og organisasjonell. 

De personlige er å utvikle meg selv til en bedre leder og få avdelingen min til å fungere godt, samt å legge grunnlaget for etterfølgeren min. Jeg ønsker også å bli bedre på sammspillet mellom avdelinger og forbedre kommunikasjonen og rutinene på ledernivået i organisasjonen.

De organisasjonelle målsetningene går på gjennomføring og oppnåelse for avdelingen. Jeg har som målsetning at avdelingen skal ha klart å få organisasjonen over på ny innfrastruktur i løpet av semesteret. Dette innebærer å fjerne mange gamle systemer og migrere nødvendige systemer over på ny hardware. 

For det neste semesteret skal IT-avdelingen komme med ny versjon av alle systemene vi utvikler selv. Og vi skal forbedre produksjonsprosessen i studentmediene. Forenkling av administrasjonen i gjengen er også noe vi kommer til å jobbe med. Jo lettere det er for administrasjonen å gjøre det de skal, jo mindre utbrente blir de som er med. 

\paragraph{Gjennomføringen} skjer hovedsaklig med noen enkle tiltak. Det er fokus på det sosiale, tilrettelegge arbeidet og innformasjonsflyt. 

Fokus på det sosiale forbedrer sammholdet i gruppa og tilknytningen individene har til hverandre. Når avdelingen har en sosial tilhørighet i hverandre har man klart å skape en identitet som man kan forme inn i resten av organisasjonen. Dette er en kontinuerlig prosess som gjøres litt ved sepparering og mye med integrering. Seppareringen er til for at avdelingen skal kunne skape sin egen identitet og dermed skape en tilhørighetsfølelse for avdelinges medlemmer.
Når avdelingsidentiteten er på plass begynner arbeidet med å bygge broer inn i resten av organisasjonen. Dette skjer allerede gjennom felles dugnader og felles sosialisering på Samfundet, men fokuset vil bli større på dette når de nye har blitt litt varmere i trøya.

Tilretteleggingen av arbeidet er viktig for at folk skal få en struktur å forholde seg til. Mennesker er enkle sånn. Har man en rutine å forholde seg til vil man øke effektiviteten etter at man har blitt vandt til rutinen. Dette går hovedsaklig på faste arbeidsdager og møtetider. Fast tidspunkt, fast lokasjon og informasjon i god tid før det skjer noe gjør at folk kan planlegge. I tilretteleggingen kommer også opplæring og kursing. Det er viktig at de som er med føler de lærer noe nytt og at de prøver ting some er utenfor komforsonen, men fortsatt innafor samme fagfelt. 

Innformasjonsflyten kommer det stadig klager på i organisasjonen. Spessielt i avdelinger som har lite med de andre avdelingene å gjøre. Dette gjelder typisk informasjon på mail. Ellers har det med filtrering av informasjon å gjøre. Her er det spessielt innformasjon fra ledelsen og nedover i organisasjonen. IT har en tendens til å være i den possisjonen at veldig mye av informasojnen som kommer fra ledelsen ikke har så mye med oss å gjøre. Et annet viktig aspekt av informasjonsflyten er bevistgjøring av hvem som trenger hvilken informasjon. Samt å informere ledelsen om hva IT gjør og status på framgangen der.  

\section{Forventninger}
%(4) Hva forventer du av faget Ledelse i praksis? Finnes det noen problemstillinger som
%du synes er spesielt interessante som gjengsjef/styremedlem?, se dette opp mot temaer
%i pensumboken Teamet. Kan noe av dette brukes som emne i den avsluttende
%oppgaven (uten at du binder deg til problemstilling allerede nå)?

Av Ledelse i Praksis forventer jeg å bli opplyst om problemstillinger jeg ikke
har tenkt på selv, lære mer om praktisk ledelse og bli bedre til å samhandle
med andre mennesker.  

\paragraph{Problemstillinger} som er interessante:

\textit{Hvilken lederstil som fungerer best.}
Dette er en problemstilling som direkte berører avdelingns måloppnåelse og
dermed er veldig interessant å utforske gjennom semesteret. 

\textit{Om man skal ta på seg forskjellige roller i forskjellige sammenhenger og
eventuelt hvilken rolle i hvilken setting.}
Hovedrollene som er viktige å vurder er rollen som opponent, rollen som
rådgiver og rollen som konsulent. 

Opponent rollen er rollen hvor man stiller kritiske spørsmål og er i
opposisjon. Opposisjons possisjonen er viktig for å feilprøve nye løsninger og
kvalitetssikre beslutninger som blir tatt. 

Rådgiverrollen er til for å komme fram til mulige endringer i organisasjonen og
komme med konstruktive forslag til forbedringer. 

Konsulent rollen har til hensikt å tilby tjenester for organisasjonen. Dette
for å få avstand og en viss grad av professjonalitet i sammhendlingene innad i
organisasjonen. Veldig typisk situasjon for denne rollen er når man trenger en
ny teknisk løsning i en avdeling gitt og løsningen skal utvikles av en annen
avdeling. 

\textit{Hvilke verdier som kommer først. Skal man prioritere organisasjonens beste på
bekostning av medlemmene?}
Denne avveiningen er problematisk med tanke på måloppnåelsen. Hvis
organisasjonens beste skal prioriteres vil det i en del tilfeller gå ut over
medlemmene. Men fra den andre siden vil det også gå utover organisasjonen om
medlemmene ikke er motiverte til å gjøre sin del av jobben. Avveiningen må tas
hele tiden underveis og er viss balanse må opprettholdes. Dårlige valg kan
følge organisasjonen i lenger perioder. Det gjelder å skape den best mulige
situasjonen til sin etterfølger. 

\paragraph{Til bruk i avsluttende oppgave} er spørsmålet om rollen man har og
hvilke verdier som veier tyngs egnet som problemstillinger. Antagelig er
problemstillingen rundt roller den beste å bassere en oppgave på.  

\section{Målsetting}
%− Hvordan kan gjengen/seksjonen komme fram til sine egne målsetninger
%sammen?
Avdelings mål settes mye etter de prekære behovene som fins i organisasjonen.
Det er de viktigste problemene som blir prioritert først og så setter man et
tidsestimat for å få en målsetning. Andre målsetninger kommer gjerne fra ideer
og innspill fra avdelingen man er i. Disse innspillene gjøres ofte om til egne
eller deler av andre målsetninger.  

\section{Skape eierskap}
%− Hvordan kan man som gjengsjef/styremedlem skape eierskap til målsetningene
%hos de andre i gjengen/seksjonen?
Innad i avdelingen min er eierskap en innebyg del av prosessen. Alt vi lager og
gjør fører til at vi får sterkere eierskap til det vi gjør. Dette har ikke noen
direkte innvirkning på målene som er satt, men inndirekte vil en sterkere
eierskapsfølelse føre til mer og bedre jobbing. 

For å øke eierskapet kan man feks fokusere på det sosiale i avdelingen og skape
et bedre sosialt samhold. Man må også gi ansvar til folk for at de skal
investere i organisasjonen og dermed øke eierskapet sitt til organisasjonen. 

\end{document}
This is never printed
