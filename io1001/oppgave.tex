% latex article template

% cheat sheet(eng): http://www.pvv.ntnu.no/~walle/latex/dokumentasjon/latexsheet.pdf
% cheat sheet2(eng): http://www.pvv.ntnu.no/~walle/latex/dokumentasjon/LaTeX-cheat-sheet.pdf
% reference manual(eng): http://ctan.uib.no/info/latex2e-help-texinfo/latex2e.html

% The document class defines the type of document. Presentation, article, letter, etc. 
\documentclass[12pt, a4paper]{article}

% packages to be used. needed to use images and such things. 
\usepackage[pdfborder=0 0 0]{hyperref}
\usepackage[utf8]{inputenc}
\usepackage[english]{babel}
\usepackage{graphicx}
\PassOptionsToPackage{hyphens}{url}

% hides the section numbering. 
\setcounter{secnumdepth}{-1}

% Graphics/image lications and extensions. 
\DeclareGraphicsExtensions{.pdf, .png, .jpg, .jpeg}
\graphicspath{{./images/}}

% Title or header for the document. 
\title{
	Avsluttende oppgave, Ledelse i Praksis
}
% Author
\author{
	Magnus L Kirø \\
	IT-sjef, Studentmediene i Trondheim
}
\date{\today}

\begin{document}
\maketitle
\pagenumbering{arabic}

\begin{abstract}
Rollene man har og hvilke roller man tar på seg til hver anledning skal
diskuteres i denne oppgaven. Konflikter mellom roller og rolleinnehavere vil
være et viktig element. Samt problematikken ved å ha flere verv og roller
samtidig, og muligens med motstridende interesser. 
\end{abstract}

\section{Introduction}
This is time for all good men to come to the aid of their party!

\paragraph{Outline}
The remainder of this article is organized as follows.
Section~\ref{previous work} gives account of previous work.
Our new and exciting results are described in Section~\ref{results}.
Finally, Section~\ref{conclusions} gives the conclusions.

\section{Previous work}\label{previous work}
A much longer \LaTeXe{} example was written by Gil~\cite{Gil:02}.

\section{Results}\label{results}
In this section we describe the results.

\section{Conclusions}\label{conclusions}
We worked hard, and achieved very little.

% The following has nothing to do with structure or content. It's just nice to have easily available without googling. 
\section{Useful stuffs}

% basic table
% http://en.wikibooks.org/wiki/LaTeX/Tables
% the "{ l c r }" part decides if the content of a cell should be at the center, left or right. 
\begin{tabular}{ l c r }
  1 & 2 & 3 \\
  4 & 5 & 6 \\
  7 & 8 & 9 \\
\end{tabular}

$_This is subscript$
$^This is Superscript$

% imgae example. 
\begin{figure}[htb]
    \centering
    \includegraphics[width=\textwidth]{nameOfImageFile} 
    \caption{The text that shows under the image, image text.}
    \label{fig:FigureLableName}
\end{figure}


\bibliographystyle{abbrv}
\bibliography{main}

\end{document}
This is never printed
