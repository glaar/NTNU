% latex article template

% cheat sheet(eng): http://www.pvv.ntnu.no/~walle/latex/dokumentasjon/latexsheet.pdf
% cheat sheet2(eng): http://www.pvv.ntnu.no/~walle/latex/dokumentasjon/LaTeX-cheat-sheet.pdf
% reference manual(eng): http://ctan.uib.no/info/latex2e-help-texinfo/latex2e.html

% The document class defines the type of document. Presentation, article, letter, etc. 
\documentclass[12pt, a4paper]{article}

% packages to be used. needed to use images and such things. 
\usepackage[pdfborder=0 0 0]{hyperref}
\usepackage[utf8]{inputenc}
\usepackage[english]{babel}
\usepackage{graphicx}
\PassOptionsToPackage{hyphens}{url}

% hides the section numbering. 
\setcounter{secnumdepth}{-1}

% Graphics/image lications and extensions. 
\DeclareGraphicsExtensions{.pdf, .png, .jpg, .jpeg}
\graphicspath{{./images/}}

% Title or header for the document. 
\title{
	Møteledelse og Internkommunikasjon, Øving 2. Ledelse i Praksis.
}
% Author
\author{
	Magnus L Kirø \\
	IT-sjef, Studentmediene i Trondheim % in what capacity are you presenting this document? as yourself, the sales manager, ceo, etc? 
}
\date{\today}

\begin{document}
\maketitle
\pagenumbering{arabic}

\section{Møte deltagelse}
Til daglig deltar jeg på en del møter. Som medlem av ledelsen i studentmediene
er det to møter i uka, ledergruppemøte og driftsmøte. Også har hele
studentmediene allmøte annenhver uke. I tillegg til det har IT-avdelingen et
møte i uka. Disse 4 møtene er bare vervrelaterte møter. Jeg har også minst to
andre møter i uka som jeg burde være på. 

Erfaringsmessig har møter stort sett liten innvirkning på det gjennomsnittlige
medlem i studentmediene. Derfor er møter mye mer en sosial sammenkomst enn noe
annet. 

Ledelsesmøter har ofte agenda og referat. Deltagelse på slike møter er
hovedorganet for kommunikasjon mellom avdelinger. 

Jeg har også erfart at det er effektivt å ha stående møter. Kjappe møter hvor
man kort og greit diskuterer småting. Da får alle som er med sagt det de skal
kjapt for så å kunne gå tilbake til det de holdt på med. Møteleder bør sende ut
referat fra slike møter. Det for å ha noe å referere til i etterkant. 

Tekniske grupperinger har også en tendens til å ikke like møter. Derfor er det
gunstig å holde møtene korte, konsise, og rett på sak. 

\section{Møte elementer}
Inkalling brukes når man skal opplyse parter om at det er møte, hvor det er og
når det er. Dette er viktig for møter hvor flere grupperinger skal være
tilstede. Om man har møte innad i en gruppe så er det ikke sikkert det er
nødvendig å sende inkalling hver gang. Da kan det fungere like godt med å sette
opp en felles kalender som inneholder alle møtedatoene med tidspunkt. Dette
effektiviserer også arbeidet med å sende ut møteinkallng.  

Personlig syns jeg sakslite til møter jeg skal være med på er essensielt. Om
jeg ikke vet hva som skal foregå på møtet kan jeg ikke forberede meg til det og
dermed vil møtet bli mindre effektivt enn det kunne vært. 

Utover det så er det en gang slik at mange ikke leser møteinkallinger eller
sakspapirer. Så man må se ann gruppen man skal holde møte for og tilpasse
informasjonen som sendes ut til hver enkelt part.

Om man sender ut saksinkalling, så er det ikke sikkert det er nødvendig med et
møte engang. Hvis det bare er enveis informasjon, så vil sakspapiret eller
møtet være overflødig. Enten leser folk sakspapiret, eller så kommer folk på
møtet. 

Sendes det ut en god saksliste i forkant av møtet er det lett for folk å se om
de trenger å være der eller ikke. 

Møtereferater er essensielle. Fins det ikke et møtereferat er det det samme som
om møtet aldri var. Er det noen som ikke var på møtet må de ha muligheten til å
lese seg opp på beslutninger som taes, forbedringer i organisasjonen og
problemer som har oppstått siden sist. 

Møtereferater må også sendes ut kjapt. Typisk samme dag eller senest dagen
etterpå. 

Om det ikke fins et referat fra et møte jeg har vært på så har jeg kastet bort
tiden min fulstendig ved å være på det møtet. 

Når man vedtar noe, så kan det gjøres på flere måter. Håndsoprekning,
aklamasjon eller andre fungerende metoder. 

Det som ofte skjer er at noen har tekt på noe lurt og presenterer saken og
mulig løsning. Om ingen har noe imot det eller har noen innvendinger blir det
ofte vedtatt på stillhet. Den som tier samtykker. 

Man tager hva man haver og putter i en gryte. Også kjent som at man benytter
seg av det som er ledig. Det å finne møterom for forskjellige grupper er en
utfordring. Helst vil man ha det samme rommet her gang, det må være god nok
luft, helst projektor, stort nok og være på et sted som folk flest finner fram
til. Det er ofte møteleder som finner rommet man skal være i. 

\section{Møteforberedelser}
Som forberedelser til møter leser jeg over innkallingen, sakspapirer og
saksliste før møtet. Dette gjør at jeg letter kan tenke meg til det som skal
skje på møtet, og hvilke innspill jeg har å komme med. 

Om møtet jeg er innkalt til halder om en bestemt til, så er det naturlig at jeg
setter meg inn i saken det gjelder og forbereder innspill. 

Før møter snakker jeg stort sett ikke med noen om det jeg må forberede til
møter. Jeg har som oftest kontroll på det jeg må presentere i møter og det
innholdet mine møter har. Møter jeg kaller inn til har ofte den hensikten at
man skal finne ut noe eller løse et problem. Derfor trenger man ikke snakke med
så mange parter før man kaller inn til et møte. 

Det er veldig få møter jeg kaller inn til for å finne ut noe. Jeg kaller oftest
inn til informasjonsmøter, hvor jeg informerer andre. Eller at jeg har kalt inn
flere parter slik at partene kan utveksle informasjon seg i mellom. 

Agendaen for møter blir satt av møtelederen. Ihvertfall før møtet. Om man ser
at det trengs endringer i agendaen underveis i møtet må man ta det som det
kommer og tilpasse møtet.

Det er vanskelig å si om beslutninger blir tatt utenfor møter i Studentmediene.
Det er fordi det er en stor organisasjon. I min avdeling kan jeg si at en del
beslutninger blir tatt utenfor møter, men det er ingenting som tilsier at
beslutningen ikke kunne vært tatt i et møte. Hovedgrunnen til at beslutninger
tas utenfor møter er at mange ikke bryr seg om alle beslutningene. De vet ikke
om dem og har derfor ikke gjort seg opp noen mening om det heller. 

I en pragmatisk avdeling som IT-avdelingen er det aldri noen problemer med
beslutninger. Vi velger den beste løsningen uansett. Om vi bruker litt tid på å
diskutere muligheten og meninger for og imot så er det helt griet. Vi
diskuterer for å finne den beste løsningen. 

Beslutniger som tas utenfor møter effektiviserer organisasjonen. Det gjør at
ting blir bedre fortere. Det er en gang slik at de som tar beslutninger ofte er
kompetente til å ta den gitte beslutningen. Noe som gjør at man i stor grad
ikke tar dårlige beslutninger. 

Om det blir tatt dårlige beslutninger, så er det bare å revurdere den
beslutningen og forbedre arbeidsprosessen og organisasonen litt. 

\section{Møter som Teambygging}
I studentmediene er møter hovedarenaen for sosial tilhørighet. Møter  i den
tradisjonelle kapasiteten brukes under sosiale møter for å informere om hva som
skjer i organisasjonen. Møtet i sin helhet er til for at folk skal føle en
tilhørighet, føle at de er med på noe, at de har eierskap til noe. 

I IT-avdelingen har vi samkjørt møter og arbeidsdager. Vi har møte og
arbeidsdag hver onsdag. Da jobber folk med arbeidsoppgavene sine og er sosiale
samtidig. Man diskuterer problemer og snakker om ting som har skjedd. Den
tradisjonelle møtebiten blir holdt kort for at folk skal kunne holde på med det
de vil. Den generelle tilbakemeldingen er at folk syns det er bra med korte
møter med kjapp informasjon. 

\section{Internkommunikasjon i sammenheng med møter}
Møtene styrer i stor grad den interne kommunikasjonen i organisasjonen. Dette
har veldig mye med hvem som er på møter, hvem som har sentrale roller i
organisasjonen og hvem som er kjent i mengden. 

Om en sentral person ikke er på møtet, eller noen møter etterhverandre, så vil
resten av organisasjonen føle at den gitte avdelingen som ikke er representert
ikke gjør noe, selv om de kanskje gjør mer enn andre. 

Møtedeltagelsen er essensiell for at internkommunikasjonen skal fungere. Og for
at organisasjonen skal funger.  

\end{document}
This is never printed
