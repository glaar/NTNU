% latex article template

% cheat sheet(eng): http://www.pvv.ntnu.no/~walle/latex/dokumentasjon/latexsheet.pdf
% cheat sheet2(eng): http://www.pvv.ntnu.no/~walle/latex/dokumentasjon/LaTeX-cheat-sheet.pdf
% reference manual(eng): http://ctan.uib.no/info/latex2e-help-texinfo/latex2e.html

% The document class defines the type of document. Presentation, article, letter, etc. 
\documentclass[12pt, a4paper]{article}

% packages to be used. needed to use images and such things. 
\usepackage[pdfborder={0 0 0},colorlinks=true,linkcolor=blue]{hyperref}
\usepackage[utf8]{inputenc}
\usepackage[english]{babel}
\usepackage{graphicx}
\PassOptionsToPackage{hyphens}{url}

% hides the section numbering. 
% this makes \ref{marker} show up as empty. use \nameref{}, or \pageref{}
%\setcounter{secnumdepth}{-1}

% Graphics/image lications and extensions. 
\DeclareGraphicsExtensions{.pdf, .png, .jpg, .jpeg}
\graphicspath{{./images/}}

% Title or header for the document. 
\title{
	Motorikk, persepsjon, og kognisjon \\ hos barn under to år. 
}
% Author
\author{
	Magnus L. Kirø \\
	%Representing ThisOrThat organization % in what capacity are you presenting this document? as yourself, the sales manager, ceo, etc? 
}
\date{\today}

\begin{document}
\maketitle
\pagenumbering{arabic}

%\begin{abstract}
%1000 ord, +-100. 
%
%(motorikk vs persepsjon)
%
%Drøft mulige sammenhenger mellom motorikk og persepsjon/kognisjon hos barn i
%deres første to leveår.
%
%Teksten bør problematisere tema og ikke bare gjengi pensum. 
%
%\end{abstract}

\tableofcontents
\newpage

\section{Introduksjon} %150
%tema: motorikk og persepsjon/kognisjon i barn mellom 0 og 2 år.

Oppgavens fokus er barn i alderen 0 til 2 år, og sammenhengen mellom kognisjon
og motorikk.  

Fra psykologien har Gibson og Piaget vært sterke bidragsytere innen persepsjon
vår forståelse av den. 

Utviklingen av motorikken, og kognisjon, hos barn vil bli belyst. 
Barns motoriske ferdighter vil bli satt i sammenheng med barns persepsjon. 
Og problematisering av dette vil oppsummere oppgaven.  

\section{Motorikk i utviklingen} %250 %a275

I tidlig utvikling hos barn kan man observere en rask læring og tilpasning til
omgivelsene. Barn utvikler seg fra lett rulling fra side til rygg eller mage,
og over til å gå og løpe etter to år. Utviklingen har kjapt gått fra
grunnleggende bevegelse til å utøve avansert motorikk, som løping. 

I det første året lærer barn enkle ting. Håndbevegelser som griping og peking
utvikler seg raskt. Krabbing kommer sakte gjennom det første året og barnet kan
begynne å stå mot slutten. Barn utvikler krabbing ved først å løfte overkroppen
med armene, og trekke beine inn under kroppen. Til å begynne med er det lite
bevegelse frammover, men det kommer raskt. Krabbing opp og ned trapper er mulig
når barnet nærmer seg ett år.  
Utforsking av objekter blir en fremtredende
aktivitet. Ting som stabling og prikking av objekter er typiske
utforskningshandlinger. 

Utover bevisste handliger utvikler barn også reflekser. I det første året er
det reflekser som gripereflekse, blunking, og suging som er markante.  

I løpet av det andre året, til barnet blir to år gammelt, utvikles krabbingen
fra å være utforskende til å være kjapp og dyktig. Balansen er i utviklig, og
barnet klarer å reise seg og stå på egenhånd. Mot slutten av perioden kan
barnet gå alene. Barnet har dog vanskeligheter med hindre, og faller ofte. 
Gåing ved hjelp av leker, dytting og draing, er normalt. Løping kommer gradvis,
men barnet har problemer med å stoppe. 

Tegning blir en del av hverdagen, og barnet kan bære leker fra et sted til et
annet. I løpet av det andre året begynner barnet å hjelpe til med å spise selv,
holding av skjeer og glass/kopper.  

\section{Kognisjon i utviklingen} %250 %a315
Sentralt i utviklingen av kognisjon står det visuelle, hørsel, smak, og
berøring. Det er berøring og det visuelle som er tettest knyttet til
motorikken. Tale og sosiale ferdigheter ligger også til den kognitive
utviklingen.   
% y1
I det første året observeres det at barn utvikler fokus for objekter og voksne.
Barnet vil følge lyd, og se i rettningen lyder kommer fra. Visuelt utvikler
barnet evnen til å oppfatte dybde, og evnen til å følge blikket til andre.
Barnet vil også se etter leker som er sluppet, samt å følge med på hvor leker
som blir sluppet havner.   

Talemessig begynner barn å babble mot foreldrene og andre voksne de kjenner
igjen. Barn begynenr å like det å lage lyder, og kan etterhvert lage lyder,
kombinasjoner av vokaler og konsonanter, som 'mumu' og 'dada'. Snart etter de
sammensatte lydene kommer repetisjon av ord i rekkefølge, gjerne 2 og 3  ord
etterhverandre. 

De sosiale utviklingene går i hovedsak på at barn smiler til sine foreldre og
at de liker å se på nye ansikter. Utover det kan barn vise angst overfor
fremmede og bli skremt av brå lyder. Enkle kommandoer blir forstått, og barnet
hjelper til med å kle på seg selv.  

%y2
I det andre året ser man rask utvikling på objektforståelse. Objekter kan
flyttes fra en hånd til en annen når barnet får et nytt objekt. Og barnet kan
håndtere opp til fire objekter samtidig, ved å sette fra seg et på fanget e.l. 
Billedbøker blir interessante, formleker blir gjennomførbare og man kan se enkle
ansiktsuttrykk. Autisme blir som oftest avdekket i denne alderen. 

Barnet kan gjenkjenne favorittsangen sin. 
Talen hos ett-åringer utvikler seg til å ha flere ord, og korte setninger på to
til tre ord. 
På den sosiale fronten kan barnet nå delta i paralell lek, det at to barn kan
leke sammen uten å forstyrre hverandre. Barnet begynner også å spise på
egenhånd. 

\section{Sammenhenger} %250
Barn som nærmer seg to‐års alderen er i stand til å forestille seg
handlinger og objekter uten å faktisk utføre handlingene rent fysisk.
Dette danner grunlag for at det motoriske henger sammen med det kognitive via
persepsjoner. Persepsjonene vil være en form for kommunikasjonskanal mellom den
oppfattelsen mennesker har av verden og det vi kan sanse av omgivelsene. 

Piaget deler utviklingen opp i stadier. Det første stadiet er det
sensomotoriske, hvor barn opp til 24 måndter utvikler reflekser, coordinasjon,
og mental representasjon. I dette stadiet mener Piaget at barn utvikler
forståelse og kunnskap om verden, på en progressiv måte, ved å samkjøre fysiske
interaksjoner med erfaringer. Her blir den mentale representasjonen av verden
aktivt tilpasset av de nye intrykkene som blir erfart gjennom sansing og
berøring. 

En Gibsoniansk tilnærming til utviklnigen hos barn er å betrakte
muligheter i miljøet og motivasjonen til å utforske dem. Dette er nært knytte
til teorien om direkte persepsjon fra Gibson. Barn utvikler dybdesynet, og
dermed finner de flere muligheter i verden, noe som forsterker ønsket om å
utforske verden mer. Dette er en forsterkende sirkel.  
Hovedfaktoren er at barn lærer, av nødvendighet, å oppfatte informasjon, fra forskjellige
objekter, som kan benyttes i daglige aktiviteter.  

I essens er disse to tilnærmingene adskilt av gruppering versus kontinuitet.
Piaget skiller utviklingen i stadier, og har plassert forskjellige ferdigheter
til forskjellige tider av utviklingen. Mens Gibson-tradisjonen setter
utviklingen i en slevforsterkende syklus, bedre dybdesyn gir mer utforsking som
gir mer dybdesyn. For Piaget er det utviklingen av den kognitive oppfattelse av
verden som driver utforskningen vidre. Mens Gibson-vinklingen setter de
motoriske sanseintrykkene som en sterkere pådriver for utvikling. 

\section{Problematisering} %150 
Både Piaget og Gibson-tradisjonen har blitt kritisert gjennom årene.
Gibson-tradisjonen for å være uklar når det kommer til kognisjon. Direkte
persepsjon står sentralt, mens indirekte og forstående kognisjon blir oversett.   
En del av forsøkene fra Gibson-tradisjonen er også blitt kritiser for ikke å ta
for seg voksne, man får da ikke sette på virkningene over et større
aldersspenn.  

For Piaget sin del går kritikken mye ut på at man ikke kan dele opp utviklingen
i klare stadier. Det er for mye overlapp og gradvis utvikling til at stadiene
gir god gjengivelse av den observerte utviklingen. I noen tilfeller blir også
barn undervurdert. Det er også ting som tyder på at utviklingen er
domenespesifikk, noe Piaget ikke tar hensyn til.  

Et samspill mellom motorikk, persepsjon, og kognisjon kan anses for å være
naturlig. Det er vanskelig å si at en av de tre aspektene har økt innflytelse
over de andre, og det er heller sansynlig at de tre aspektene påvirker
hverandre i ulik grad.  

%--notater--------------------------------------------%

%piaget: stadieteori om kognitiv utvikling
%Kohlberg: stadieteori om moralsk utviklnig. 
%Freud: stadieteori om psykoseksuell utvikling
%Eriksson: stadieteori om psykososial utvikling

%# Eriksson teori om psykososial utvikling: 
%tillit vs mistillit:(0-12mnd) 
%Energi og håp. Barnet utvikler
%tillit til omgivelsene i forhold til å
%få tilfredsstilt sine
%grunnleggende behov. 
%
%Autonomi vs skam og tvil: (13-24 mnd) 
%Selvkontroll og viljestyrke.
%Barnet lærer å kontrollere
%omgivelsene og utvikler en
%opplevelse av å ha fri vilje. Det
%utvikler også følelser som skam
%og tvil i tilknytning
%til “feil” bruk
%av vilje og kontroll.
%
%#John Bowlby mente å observere at behovet for moderlige omsorg hos mennesker er
%like presserende som behovet for mat og vann. Han mente at tilknytningsatferd
%hos mennesker må  betraktes som et medfødt atferdssystem. Bowlby  beskrev to
%former av tilknytningsatferd hos barn: Signalatferd (ansiktsuttrykk, gråt,
%gester, kroppsholdning)  Tilnærmingsatferd (rulling, krabbing) 
%
%Vygotskij: Barnet er fra fødselen av innstilt på samspill med andre mennesker. Barnet tar i bruk kulturens språk og tenkemåter og gjør dem til sine – Vygotskij kaller det internalisering. 
%\section{Motorikk}
%Grovmotorikk – bevegelser med kropp, armer og ben. Finmotorikk – bevegelser i hender, fingrer og tær.
%
%To prinsipper for motorisk utvikling: Fra hodet og nedover mot tærne
%(cephalocaudal) Fra kroppens midtlinje og utover mot ytterkantene
%(proximodistal). 
%
%Utviklingsreflekser er bevegelser som fins i den tidlige
%spedbarnsalderen, men som så forsvinner. 
%
%Griperefleks: Fingrene bøyer seg sammen når barnets håndflate blir berørt.
%
%Asymmetrisk tonisk nakkerefleks: Når barnets hode er dreid til siden vil arm og
%ben på den siden som hodet vender mot strekke seg ut. 
%
%Gangrefleks: Nyfødte barn gjør gåbevegelser med bena når de blir holdt oppreist
%og fotsålene blir utsatt for litt trykk.
%
% grov motorisk utvikling, se screenshot. (ca 20150226-1422) 


% references / bibliography
%\cite{deyer}
%\addcontentsline{toc}{section}{References}
%\begin{thebibliography}{1}
%\bibitem{deyer}{
%Dyer, J. H., Kale, P., & Singh, H. (2001). Strategic alliances work. MIT Sloan
%Management Review, 37-43.}

%\bibitem{deyer}{
%Dyer, J. H., Kale, P., & Singh, H. (2001). Strategic alliances work. MIT Sloan
%Management Review, 37-43.}

%\end{thebibliography}


\end{document}
This is never printed
